\documentclass{ctexart}
\usepackage[T1]{fontenc}
\usepackage{geometry}
\usepackage{listings}
\usepackage{hyperref}
\usepackage{graphicx}
\usepackage{subfigure}
\usepackage{parskip}
\usepackage{indentfirst}
\usepackage{titlesec}
\usepackage{mathtools}
% \usepackage{ctex}
\graphicspath{{./images/}}
% \geometry{a4paper,left=1.25cm,right=1.25cm,top=2cm,bottom=2cm}
\setlength{\parindent}{20pt}
\titlespacing*{\paragraph}{20pt}{1em}{1em}

\title{NIS4307 人工智能导论大作业}
\author{
    孙靖峰\\
    \and
    蒋竺君\\
    \and
    赵子辰
}
\date{编辑于\today}


\begin{document}

\begin{titlepage}
    \maketitle
\end{titlepage}

% \tableofcontents
\newpage

\section{项目背景}

在信息爆炸的社交媒体时代,谣言的传播速度和影响力呈指数级增长。虚假新闻、不实信息在社交平台(如微博、Twitter、Facebook)上的泛滥,不仅误导公众认知,还可能引发社会恐慌、影响金融市场稳定,甚至威胁公共安全(如公共卫生事件中的谣言)。据《麻省理工学院技术评论》研究,虚假信息在社交媒体上的传播速度比真实信息快6倍,且具有更强的用户参与度。因此,自动化谣言检测成为互联网内容治理的关键技术需求。

\section{模型介绍}

\subsection{逻辑回归(Logistic Regression)}

逻辑回归是机器学习分类问题中最简单、最基础的模型。它假设数据服从伯努利分布,通过sigmoid函数将线性回归的输出映射到$[0,1]$区间,表示样本属于某个类别的概率。\par
逻辑回归的概率计算公式为:
\begin{equation}
    P(y=1|x) = sigmoid(w^T x + b) =\frac{1}{1+e^{-(w^T x+b)}}
\end{equation}\par
其中,$w$为权重,$b$为截距。Sigmoid函数可以将任意实数压缩到$(0,1)$区间,使输出结果有实际意义。

逻辑回归使用交叉熵损失(Cross-Entropy Loss)函数,以避免线性回归中的均方误差导致的非凸优化问题:
\begin{equation}
    J(w) = -\frac{1}{m} \sum_{i = 1}^{m} [y^{(i)} \log(h_w (x^{(i)})) + (1-y^{(i)})\log (1-h_w (x^{(i)}))])
\end{equation}\par
其中,$m$为样本数量,$y^{(i)}$为真实标签,$h_w (x^{(i)})$为预测概率。\par
在计算过程中,使用梯度下降法最小化损失函数:
\begin{equation}
    w_{new} = w - \alpha \cdot \nabla_w J(w)
\end{equation}\par

\subsection{高斯回归(Gaussian Regression)}

高斯回归假设每个类别的数据都服从多元高斯分布:
\begin{equation}
    p(x | y = k) = \frac{1}{(2 \pi)^{d/2} |\Sigma_k|^{1/2}} \exp (-\frac{1}{2} (x - \mu_k)^T \Sigma^{-1} (x-\mu_k))
\end{equation}\par
其中,$\mu_k$为第$k$类的均值向量,$\Sigma_k$为第$k$类的协方差矩阵,$d$则是特征维度。\par
后续进行决策时,先通过贝叶斯定理计算样本$x$属于类别$k$的概率:
\begin{equation}
    p(y = k | x) = \frac{p(x | y = k)p(y = k)}{\sum_{l} p(x| y = l)p(y = l)}
\end{equation}\par
其中,$p(y = k)$是类别的先验概率,通常用训练集中各类别的比例估计。\par
然后再选择使后验概率最大的类别:
\begin{equation}
    \hat{y} = \arg \max_k p(y = k | x)
\end{equation}

\section{数据处理与模型训练}

\begin{figure}
    \centering
    \includegraphics[width=0.7\linewidth]{training_set.png}
    \caption{\label{training_set} 训练集的内容}
\end{figure}

如图\ref{training_set}所示,数据集中有许多URL、特殊符号、用户名、标签等,因而我们需要进行一系列预处理,然后再训练模型。

\subsection{文本预处理}

首先,我们需要将所有文本转为小写,然后我们移除URL、用户、话题标签、标点符号和数字,只保留文本内容。




\end{document}

